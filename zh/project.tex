%----------------------------------------------------------------------------------------
%	SECTION TITLE
%----------------------------------------------------------------------------------------

\cvsection{Projects}

%----------------------------------------------------------------------------------------
%	SECTION CONTENT
%----------------------------------------------------------------------------------------

\cvsubsection{\textbf{Graph Systems for Hybrid Workload}}

\vspace{1mm}

\begin{cventries}

%------------------------------------------------

\cventry
{开发者} % Affiliation/role
{Vegito+G: 大规模关系型事务与图分析混合处理引擎} % Organization/group
{智能计算实验室,达摩院,阿里巴巴} % Location
{2021.10 - 至今} % Date(s)
{ % Description(s) of experience/contributions/knowledge
	\begin{cvitems}
		  \item {本项目的目标是构建一个松耦合的实时图分析系统(OLAP),系统能够实时消费现有关系型数据源(OLTP)提供的数据;}
		  \item {我在本项目中负责设计关系型模型到图模型的映射接口,并构建高性能动态图存储及实时分析引擎;}
		 \item {经过测试,系统相比现有Graph HTAP系统LiveGraph在图计算负载(例:PageRank,SSSP)获得最高\emphstyle{2.2倍}的性能提升。}
	\end{cvitems} 
}

\vspace{-3mm}

\cventry
{作者} % Affiliation/role
{Graphy: 面向流水线式图应用的大规模处理系统} % Organization/group
{并行与分布式系统研究所,上海交通大学} % Location
{2020.6 - 至今} % Date(s)
{ % Description(s) of experience/contributions/knowledge
	\begin{cvitems}
	 \item {本项目为蚂蚁金服合作项目,目标是构建一个面向流水线式图应用(例:Gremlin查询+图分析)的大规模图处理系统;}
	 \item {独立完成了该项目的系统设计到实验测试,从数据转换与数据移动两方面减少流水线图应用中的巨额开销;}
	 \item {测试典型流水线应用时延,Graphy相比拼接现有图系统方案(Wukong+Gemini)能够获得\emphstyle{41倍}的性能提升;相比传统分布式数据处理系统(Spark)能够获得\emphstyle{59倍}的性能提升。}
	\end{cvitems}
}

\end{cventries}

\vspace{-2mm}

\cvsubsection{\textbf{Graph Systems for Specific Workload}}

\vspace{1mm}

\begin{cventries}

\cventry
{作者} % Affiliation/role
{Wukong+G: 基于GPU加速的大规模RDF查询系统} % Organization/group
{并行与分布式系统研究所,上海交通大学} % Location
{2020.12 - 2021.10} % Date(s)
{ % Description(s) of experience/contributions/knowledge
	\begin{cvitems}
		\item {本项目使用GPU来加速大规模知识图谱存储上的RDF查询负载,并通过多查询合并技术提升多任务场景下的系统吞吐;}
		\item {独立完成项目系统实现及投稿,项目论文“Wukong+G: Fast and Concurrent RDF Query Processing Using RDMA-Assisted GPU Graph Exploration.”已经被\emphstyle{TPDS'2022}接收。}
	\end{cvitems}
}

\vspace{-3mm}

\cventry
{开发者} % Affiliation/role
{FlexGraph: 大规模图神经网络训练系统} % Organization/group
{智能计算实验室,达摩院,阿里巴巴} % Location
{2019.12 - 2020.10} % Date(s)
{ % Description(s) of experience/contributions/knowledge
	\begin{cvitems}
	 \item {本项目构建了一个高效的大规模图神经网络训练框架,面向新兴的GNN算法提出了表达性更好的NAU抽象模型;}
	 \item {我在本项目中负责原型系统构建及分布式测试,使用PowerGraph构建了分布式GNN训练系统原型;}
	 \item {项目论文“FlexGraph: A Flexible and Efficient Distributed Framework for GNN Training.”已经被\emphstyle{EuroSys'2021}接收。}
	\end{cvitems}
}

\vspace{-3mm}

\cventry
{作者} % Affiliation/role
{HyperCache: 基于超图索引加速的RDF查询系统} % Organization/group
{并行与分布式系统研究所,上海交通大学} % Location
{2021.10 - 至今} % Date(s)
{ % Description(s) of experience/contributions/knowledge
	\begin{cvitems}
	 \item {使用多元关系缓存来提升RDF查询负载的性能,为现有RDF查询系统构建了一套可扩展的分布式缓存方案;}
	 \item {提出基于关系树的缓存元数据存储方案,相比现有的Bliss标签方案在缓存匹配阶段获得\emphstyle{10倍}的性能提升;}
	 \item {经过测试,这套解决方案可以在有限的内存空间(1GB)下为端到端查询提供\emphstyle{最高15倍}的性能提升。}
	\end{cvitems}
}

\vspace{-3mm}

\cventry
{作者} % Affiliation/role
{Gemini-L: 基于混合Push/Pull计算模型的图计算系统} % Organization/group
{并行与分布式系统研究所,上海交通大学} % Location
{2020.4 - 2020.6} % Date(s)
{ % Description(s) of experience/contributions/knowledge
	\begin{cvitems}
	 \item {本系统是实现Graphy中分布式图计算引擎时衍生的想法:通过混合传统图计算中的Push/Pull模型来减少网络通讯;}
	 \item {经过测试,该系统能够在PageRank负载下最高减少\emphstyle{40\%}网络传输数据量,相比Gemini系统获得了\emphstyle{1.3倍}的性能提升;}
	 \item {该项目已在github开源:https://github.com/EsdeathYZH/Gemini-L。}
	\end{cvitems}
}

%%------------------------------------------------

\end{cventries}