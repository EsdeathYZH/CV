%----------------------------------------------------------------------------------------
%   SECTION TITLE
%----------------------------------------------------------------------------------------

\cvsection{Other Experiences}

%----------------------------------------------------------------------------------------
%   SECTION CONTENT
%----------------------------------------------------------------------------------------

\begin{cventries}

%------------------------------------------------

% FIXME
% \item {\textbf{ThreadScope}:图形界面Haskell程序 profiling工具。通过使用更高效的数据结构提升性能,内存占用降低\emphstyle{$\mathbf{40\%}$}}

\cventry
{课题组开发负责人} % Affiliation/role
{\textbf{国家重点科研专项:构建大规模时序超图系统}} % Organization/group
{https://github.com/SJTU-IPADS/wukong-cube} % Location
{2021.4-至今} % Date(s)
{ % Description(s) of experience/contributions/knowledge
    \begin{cvlargeitems}
        \item {本项目由北京理工大学牵头,涉及华中科技大学、上海交通大学、交通银行等\emphstyle{十余家单位}共同协作完成;}
        \item {我作为整个项目课题二(高效时序超图查询系统)的\emphstyle{team leader},带领三位同学共同开发面向时序超图的大规模查询系统;}
        \item {构建了面向时序超图(多元关系)模型的大规模知识图谱存储系统,目前项目仍在测试投稿阶段}
    \end{cvlargeitems}
}

\vspace{-4mm}

\cventry
{贡献者} % Affiliation/role
{Oneflow} % Organization/group
{https://github.com/Oneflow-Inc/oneflow} % Location
{2021.6 - 2021.10} % Date(s)
{ % Description(s) of experience/contributions/knowledge
    \begin{cvlargeitems}
        \item {Oneflow是一个开源的工业级分布式深度学习框架,出于对深度学习框架的兴趣,我参与到Oneflow的开源社区工作中}
        \item {为Oneflow的推理Session开发了一套C++ API,并完成了部分算子的重构工作,为Github项目提交了3个PR}
    \end{cvlargeitems}
}

\vspace{-4mm}

\cventry
{参赛队员} % Affiliation/role
{\textbf{九坤量化模拟交易撮合系统大赛}} % Organization/group
{https://github.com/EsdeathYZH/PPIW-Trading} % Location
{2022.1 - 2022.3} % Date(s)
{ % Description(s) of experience/contributions/knowledge
    \begin{cvlargeitems}
        \item {在该比赛中,我与其他两位队友在\emphstyle{两周时间内}构建了一个分布式的交易撮合系统(2Trader+2Exchange)}
        \item {我们的队伍在\emphstyle{80+参赛队伍}中脱颖而出,并在最终的决赛获得\emphstyle{季军}。}
    \end{cvlargeitems}
}

\vspace{-4mm}

\cventry
{Reviewer} % Affiliation/role
{TKDE期刊Reviewer} % Organization/group
{} % Location
{2022.3} % Date(s)
{ % Description(s) of experience/contributions/knowledge
	\begin{cvitems}
		\item {我曾有机会成为\emphstyle{CCF-A类期刊TKDE}(IEEE TRANSACTIONS ON KNOWLEDGE AND DATA ENGINEERING)的一名\emphstyle{Reviewer},参与评审了一篇投稿的文章}
	\end{cvitems}
}

\vspace{-3mm}

\cventry
{开发者} % Affiliation/role
{ByteCamp 2021} % Organization/group
{https://github.com/EsdeathYZH/ByteCampGNNSys} % Location
{2020.8} % Date(s)
{ % Description(s) of experience/contributions/knowledge
    \begin{cvlargeitems}
        \item {报名参加了字节跳动2021暑期夏令营,入选分布式存储方向;}
        \item {在\emphstyle{一周的时间}内,开发出一个大规模分布式图采样系统,在三台机器上提供\emphstyle{624批量/秒(批量大小1024)}的采样吞吐.}
    \end{cvlargeitems}
}

\vspace{-4mm}

\cventry
{课程助教} % Affiliation/role
{本科生课程助教} % Organization/group
{https://ipads.se.sjtu.edu.cn/courses/compilers/2019} % Location
{2019-2020} % Date(s)
{ % Description(s) of experience/contributions/knowledge
    \begin{cvlargeitems}
        \item {分别担任计算机系统工程(2020秋)、编译原理(2019秋)两门课程的助教}
        \item {负责作业发布及课程lab评分,在编译原理课程中使用C++重构了Tiger语言编译器作为课程lab.}
    \end{cvlargeitems}
}

\vspace{-4mm}

\cventry
{贡献者} % Affiliation/role
{其他开源项目中的代码贡献} % Organization/group
{} % Location
{2020 - 至今} % Date(s)
{ % Description(s) of experience/contributions/knowledge
	\begin{cvitems}
		\item {\textbf{Euler}:主流开源GNN训练系统。修复采样操作在Shape计算中的Bug,共贡献两个Commits。}
	\end{cvitems}
}

%%------------------------------------------------

\end{cventries}