%----------------------------------------------------------------------------------------
%	SECTION TITLE
%----------------------------------------------------------------------------------------

\cvsection{Projects}

%----------------------------------------------------------------------------------------
%	SECTION CONTENT
%----------------------------------------------------------------------------------------

\begin{cventries}

%------------------------------------------------

\cventry
{Core Member} % Affiliation/role
{Ring} % Organization/group
{XXXX Laboratory, XXXXXX University} % Location
{Mar. 2017 - PRESENT} % Date(s)
{ % Description(s) of experience/contributions/knowledge
    \begin{cvitems}
        \item {Implemented hotspot detection algorithm that can detect hotspot events from Weibo stream with latency \emphstyle{less than 1 hour}.}
        \item {Optimized the schema of Elasticsearch mappings by analyzing SlowLog, improving the time consumed by certain queries from \emphstyle{more than 1 minute to less than 1 second}.}
    \end{cvitems}
}

\vspace{-5mm}

\cventry
{Co-author} % Affiliation/role
{haskell-stdio} % Organization/group
{https://github.com/xxxxx/xxxxxx} % Location
{Mar. 2018 - PRESENT} % Date(s)
{ % Description(s) of experience/contributions/knowledge
    \begin{cvitems}
    	\item {haskell-stdio is a \textit{\textbf{personal project}} that provides high-performance containers and IO support for Haskell, receiving \emphstyle{$\mathbf{80+}$} stars on Github.}
        \item {Implemented high-performance vector library based on new compiler features, achieving \emphstyle{$\mathbf{10\%}$ ~ $\mathbf{40\%}$} improvement in performance.}
        \item {Implemented high-performance UTF-8 text via FFI bindings}
        \item {Implemented IO library based on the \textit{libuv} library, yielding \emphstyle{$\mathbf{15\%}$} higher throughput than the standard library(MIO).}
    \end{cvitems}
}

\vspace{-5mm}

\cventry
{Contributor} % Affiliation/role
{Glasgow Haskell Compiler} % Organization/group
{https://github.com/ghc/ghc} % Location
{2016.11 - PRESENT} % Date(s)
{ % Description(s) of experience/contributions/knowledge
    \begin{cvitems}
        \item {GHC is the de facto compiler of the Haskell programming language. Contributed \emphstyle{$\mathbf{30+}$} patches to GHC.}
        \item {Improved the numeric stability when generating ranges of float-point numbers.}
        \item {Fixed bug about exhaustive checking in pattern matcher.}
        \item {Optimized the performance for cases with large \texttt{.data} segment by using more efficient data structures, reducing the memory consumption from \emphstyle{GBs} to \emphstyle{MBs}.}
    \end{cvitems}
}

\vspace{-5mm}

\cventry
{Contributor} % Affiliation/role
{Contributions for other opensource projects} % Organization/group
{https://github.com/xxxxxxxxx} % Location
{2013 - PRESENT} % Date(s)
{ % Description(s) of experience/contributions/knowledge
    \begin{cvitems}
        \item {\textbf{PyTorch}: 8 commits for fixing bug about shape checking in Conv and improving the numeric stability of \texttt{linspace}, etc.}
        \item {\textbf{Apache MXNet}: 1 commit for improving the compatibility of CFFI interface for pure C compilers.}
        \item {\textbf{pandas}: 5 commits for making \texttt{ExtensionArray} support \texttt{plotting} and fixing segmentation fault caused by mistyping, etc.}
        \item {\textbf{ThreadScope}: reducing the memory usage by \emphstyle{$\mathbf{40\%}$} via leveraging more efficient data structures.}
    \end{cvitems}
}

%%------------------------------------------------

\end{cventries}
